\documentclass[a4paper,10pt,twocolumn]{article}

\usepackage[english]{babel}
\usepackage[utf8]{inputenc}
\usepackage{amsmath}
\usepackage{graphicx}
% \usepackage{ucs}
\usepackage{fontenc}
\usepackage{hyperref}
\usepackage{tikz}

\date{\today}

\title{Automatically fixing bugs in conditional statements using an SMT-Solver}

\author{
        DeMarco Favio D. \\
                Department of Computer Science\\
                Universidad de Buenos Aires
}

\begin{document}

\maketitle

\begin{abstract}
So far, the Universe is winning.
\end{abstract}

\section{Introduction}

Our contribution is a novel process that uses an SMT solver to fix conditional bugs.
 This process uses the synthesis technique of XXX (SLFP) to synthesize the correct conditions.
\begin{itemize}
 \item This process handles bugs in \texttt{if} expressions and the ternary operator.
 \item This process handles bugs of the form of missing preconditions. Nobody has ever explicitly addressed this defect class before.
 \item This process supports mainstream object-oriented programming languages and esp. the use of side-effects free unary method calls. In contrast, SemFix only supports arithmetic and boolean expressions in C code.
 \item This process is more lightweight than SemFix because it does not require the use of symbolic execution. One needs less technology and less expertise to implement and run the repair approach.
 \item Our approach is evaluated against real bugs and real test suites. Unlike SemFix which presented the repair of artificial and small-scale bugs, we are able to fix real-scale issues.
\end{itemize}

In addition, we describe a generation pattern for easily feeding an SMT solver in our problem domain.

\paragraph{Outline}
The remainder of this article is organized as follows.
Section~\ref{background} gives account of previous work.
Our new and exciting results are described in Section~\ref{evaluation}.
Finally, Section~\ref{conclusions} gives the conclusions.

\section{Background}
\label{background}

\subsection{What are conditional bugs?}

\subsection{What are missing precondition bugs?}

\subsection{SMT-based Program Synthesis}


\section{Contribution}

\subsection{Overview}

\begin{tikzpicture}[
  font=\sffamily,
  every matrix/.style={ampersand replacement=\&,column sep=2em,row sep=2em},
  box/.style={rounded corners,fill=gray,path fading=south},
  shadedbox/.style={rounded corners,fill=black!90},
  flow/.style={->, very thick},
  fail/.style={->, thick, dashed},
  every node/.style={align=center}]
  
% Position the nodes using a matrix layout
\matrix{
    \node[shadedbox] (faultLocalization) {Fault \\ Localization};
      \& \node[box] (oracleInquisition) {Oracle \\ Inquisition};
      \& \node[box] (collectionOfTestExecutionData) {Collection \\ of Test Execution \\ Data};
      \& \\

    \node[shadedbox] (sourceCodeSynthesis) {Source \\ Code \\ Synthesis};
      \& \node[box] (smtSolving) {SMT \\ Solving};
      \& \node[box] (constraintsGeneration) {Constraints \\ Generation}; \\
  };

% Draw the arrows between the nodes and label them.
\draw [flow] (faultLocalization) -- (oracleInquisition);
\draw [flow] (oracleInquisition) -- (collectionOfTestExecutionData);
\draw [flow] (collectionOfTestExecutionData) -- (constraintsGeneration);
\draw [flow] (constraintsGeneration) -- (smtSolving);
\draw [flow] (smtSolving) -- (sourceCodeSynthesis);

\draw [fail] (oracleInquisition.south) .. controls (down:.5em) and (left:8em) .. (oracleInquisition.west);
\draw [fail] (collectionOfTestExecutionData) .. controls (down:2em) and (left:9em) .. (oracleInquisition.west);
\draw [fail] (smtSolving.north) .. controls (down:2em) and (left:10em) .. (oracleInquisition.west);
\end{tikzpicture}

\label{flow_graph}

\subsection{...}

\section{Evaluation}
\label{evaluation}

\section{Conclusions}
\label{conclusions}

\bibliographystyle{abbrv}
% \bibliography{main}

\end{document}
