\documentclass{beamer}
% \documentclass[handout]{beamer}
% \setbeameroption{show notes}

\usecolortheme[named=blue]{structure}

\mode<presentation>
{
  \usetheme{Warsaw}
  \setbeamercovered{transparent}
}

\usepackage{listings}
\lstset{basicstyle=\scriptsize}

\title
  [SemFix: Program Repair via Semantic Analysis]
  {SemFix:~Program~Repair~via~Semantic~Analysis}
\author[DeMarco]{Favio~DeMarco}
\institute[U.B.A. - INRIA]{Universidad de Buenos Aires - INRIA}
\date[04/23/2013]{April 23th, 2013}
\subject{Computational Sciences}

\begin{document}

\frame
  {
    \titlepage
  }
%-----------------------------------------------------------------------------80
  \section*{Outline}
%-----------------------------------------------------------------------------80
  \frame
  {
    \frametitle{Outline of Topics}

    \tableofcontents
  }
  
  \frame
  {
    \frametitle{Semfix}
	An automated repair method based on symbolic execution, constraint solving and program synthesis.
	\note{Semfix tries to fix a bug using symbolic execution, an SMT solver and program synthesis.}
  }

  \frame
  {
    \frametitle{What?}
	\begin{description}
\item[Symbolic execution:] Analysis of programs by tracking symbolic rather than actual values. Kind of ``abstract interpretation''.
\note[item]{The program is run using variables and expressions, instead of values, taking every possible path.}

\item[Constraint solving:] The requirement on the repaired code to pass a given set of tests is formulated as a constraint.
\note[item]{Test cases are used as a source for constraints over the mentioned variables.}

\item[Program synthesis:] Automatically construct a program that provably satisfies a given high level specification in a appropriate logical calculus (instead of trying every possible combination).
\note[item]{Code is generated taking into account the mentioned constraints.}
	\end{description}
  }

  \frame
  {
    \frametitle{What we need?}
    \begin{enumerate}
     \item A program
     \item with a test suite
     \item and (at least) a failing test case.
    \end{enumerate}
      No formal specification of correct behavior is needed.
  }

  \section*{Let's start}
  \frame
  {
    \frametitle{Where is the bug?}
	\framesubtitle{Fault isolation}	
	i.e. where to fix the problem. The technique uses the ranking produced by a statistical fault
isolation tool: Tar\'antula.
\begin{equation*}
 susp(s)=\dfrac{\dfrac{failed(s)}{totalfailed}}{\dfrac{passed(s)}{totalpassed}+\dfrac{failed(s)}{totalfailed}} 
\end{equation*}
\note{The tests are run and for each executed statement a suspiciousness score is calculated based on its frequency of occurence in passing and failing executions. The techinique will try to find a fix for the program going from the most to the less suspicios statements. \textit{susp(s)} is the suspiciousness score of statement s. \textit{failed(s)} is the number of failing executions in which s occurs, and \textit{passed(s)} is the number of passing executions in which s occurs. \textit{totalfailed} is the total number of failing executions and \textit{totalpassed} is the total number of passing executions.}
  }

  \frame
  {
    \frametitle{Now what?}
    \framesubtitle{Statement-level specification inference}	

    i.e. automatically
discover the correct specification of the buggy statement.

    \note{The technique creates,
for each input to the buggy statement, the output that
would have resulted in the test passing. The program is executed \textit{concretely} with each test case input until it reaches the actual suspicios line. Then the value of the statement is replaced with a symbolic value \textit{X} and the execution continues symbolically.}
  }

  \frame
  {
    \frametitle{So?}
    \framesubtitle{Component-Based Program Synthesis}

    Given a set of input-output pairs, component-based program synthesis generates a program that satisfies all the given input-output pairs.
    \note{If
$<\alpha , \beta>$ is one of the input-output pairs, then the synthesized
program must produce output $\beta$ when its input is $\alpha$.}
    
    A set of location variables are defined for each
component (constant, minus, plus, etc.) and the synthesis process is reduced to finding
values for these location variables.
    \note{The constraint over the
location variables is in first-order logic and solved by an SMT solver.}
  }
  \section*{Methodology}
  \frame
  {
    \frametitle{The gory details}
    \begin{itemize}
     \item Only generate repairs by altering one statement.
     \note[item]{The
generated fix is always with respect to a given test suite. If all
the tests in the given test suite pass after the repair is applied,
the repair is considered to be successful.}

     \item Focus on repairs that change the right
hand side of assignments or branch predicates:
     \begin{itemize}
      \item x=$f_{buggy}(...) \rightarrow$ x=$f(...)$
      \item if($f_{buggy}(...)$) $\rightarrow$ if($f(...)$)
     \end{itemize}
     \note[item]{The expression $f(...)$ is required to be side-effect free.}
    \end{itemize}

  }
  \frame
  {
    \frametitle{Repair Constraint}
     Given a program $P$, a
test suite $T$, a repair constraint $C$ of a function $f_{buggy}$ in
program $P$ is a constraint over function $f$ such that if $f\vDash C$,
$P[f/f_{buggy}]$ passes all tests in $T$.
     \note{For each particular test case $i$, $C_i$ is a second order predicate over the function $f$. The solution of $C$ is a function $f$ which can be used to
repair the program so that all the test cases in the test suite $T$
pass.
Note that $C$ is a constraint over function $f$. Current offthe-shelf SMT solvers cannot directly solve such a formula. Component-Based Program Synthesis is used to solve this gap.}
  }
  \frame
  {
    \frametitle{Novel symbolic execution}
    
    The program is executed \textit{concretely} with input $t_i$
to statement $s$ (without
executing statement $s$). Then the result of function
$f(...)$ is seted as symbolic and \textit{symbolic execution} is started from
statement $s$.
  }
  
  \frame
  {
    \frametitle{All together now}
    \begin{description}
     \item[Input:] 
    \begin{description}
     \item[P:] The buggy program

     \item[T:] A test suite

     \item[RC:] A ranked list of potential bug root-cause
     \note{Do you remember Tar\'antula?}
    \end{description}
    \item[Output:]
    \begin{description}
     \item[r:] A repair for P

    \end{description}
     \end{description}


  }
  
  
  \begin{frame}[fragile]
    \frametitle{All together now}
    \begin{lstlisting}
while RC is not EMPTY and not TIMEOUT do

  rc = Shift(RC) // A repair candidate

  S = new Set(); // A test suite for repair generation

  Tf = ExtractFailedTests(T, P);

  while Tf.isNotEmpty() do

    S = S U Tf

    new_repair = Repair(P, S, rc)

    if new_repair == null then

      break

    end if

    P'
    ApplyRepair(P, new_repair)

    Tf = ExtractFailedTests(T, P');

  end while

  if new_repair not null then

    return new_repair

  end if

end while

\end{lstlisting}
\end{frame}

  
  \begin{frame}[fragile]
    \frametitle{All together now}
    \begin{lstlisting}
function Repair(P, S, rc)

  C = GenerateRepairConstraint(P, S, rc);

  level = 1 // The complexity of a repair

  new_repair = Synthesize(C, level);

  while new_repair == null and level <= MAX_LEVEL do

    level = level + 1

    new_repair = Synthesize(C, level);

  end while

  return new_repair

end function
\end{lstlisting}
\end{frame}
  
  \section*{Optimizations}
  \frame
  {
    \frametitle{}
  }

\end{document}
